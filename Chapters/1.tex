\documentclass[main.tex]{subfiles}

\begin{document}
    \chapter{Linear Spaces}
    A useful construct in mathematics is the \textit{linear space}. 
    Through this general notion, we can define sets abiding by certain general axioms without having to restrict ourselves to speaking about particular objects (except for the spaces themselves, of course).

    \par So in general, we have the following definition:
    \begin{definition}[Linear space]
        A \textit{linear space} is a nonempty set $V$ with two operations (called \textit{addition} and \textit{multiplication}) associated with a scalar field\footnote{for the purposes of this class, the field will be $\mathbb{R}$, and so we will use $\mathbb{R}$ for the rest of the notes. However, the complex field $\cee$ can also work, and such a linear space is called a \textit{complex linear space}.}, abiding by the following conditions or axioms:
        \begin{enumerate}
            \item (Closure under addition) For any two elements $x,y \in V$, there is a unique element in $V$ corresponding to $x+y$, called the sum of $x$ and $y$.
            \item (Closure under multiplication by real numbers) For every $x \in V$ and $a \in \mathbb{R}$ there will also be a unique element $ax \in V$ called the product of $a$ and $x$.
            \item (Commutative law) For all $x, y \in V$, we have $x+y = y+x$.
            \item (Associative law) For all $x, y, z \in V$, we have $(x+y)+z = x+(y+z)$.
            \item (Existence of a zero element) There exists an element $O \in V$ such that for all $x \in V$, $x+O = x$.
            \item (Existence of negatives) For all $x \in V$, the element $(-1)x$ exists such that $x + (-1)x = O$.
            \item (Associative law) For every $x \in V$ and all $a, b \in \ree$, we have $a(bx) = (ab)x$.
            \item (Distributive law for addition in $V$) For all $x, y \in V$ and all $a \in \ree$, we have $a(x+y) = ax+ay$.
            \item (Distributive law for addition of numbers) For all $x \in V$ and $a, b \in \ree$, we have $(a+b)x = ax + bx$.
            \item (Existence of identity) For all $x\in V$, we have $x = 1x$.
        \end{enumerate}
        We may also call a linear space a ``linear vector space'' or simply ``vector space''.
    \end{definition}
    Apostol calls axioms (1) and (2) \textit{closure axioms}, axioms (3)-(6) \textit{axioms for addition}, and axioms (7)-(10) \textit{axioms for multiplication by numbers}. 

    \subsection{Examples}
    \begin{enumerate}
        \item lol i'm way too lazy for this
    \end{enumerate}
    \section{Consequences of the axioms}
    \begin{theorem}[Zero uniqueness]
        {In any linear space $V$, there is only one zero element.}    
    \end{theorem}
    \begin{theorem}[Uniqueness of negative elements]
        {Every element $x$ in a linear space has a unique negative denoted $-x$.}
    \end{theorem}
    \begin{theorem}
        For some linear space $V$, the following properties hold for all $x,y \in V$ and $a,b \in \ree$:
        \begin{enumerate}
            \item $0x = O$.
            \item $aO = O$.
            \item $(-a)x = -(ax) = a(-x)$.
            \item If $ax = O$, then either $a = 0$ or $x = O$.
            \item If $ax = ay$ and $a \neq 0$, then $x = y$.
            \item If $ax = bx$ and $x \neq 0$, then $a = b$.
            \item $-(x+y) = (-x) + (-y) = -x - y$.
            \item $x + x = 2x, x + x + x = 3x$, and $\sum^n_{i=1} x = nx$. 
        \end{enumerate}
    \end{theorem}
    
    \begin{proof}[Proof of theorem 1]
        We are guaranteed the existence of at least one zero element by the existence axiom, so let us suppose that we have 2 of them.
        These will be denoted $O_1$ and $O_2$ respectively. 
        We simply have $$O_1 + O_2 = O_2 + O_1 = O_1 = O_2.$$
    \end{proof}
    \begin{proof}[Proof of theorem 2]
        For an element $x \in V$, suppose there are two negatives $y_1$ and $y_2$. 
        Then we have
        \begin{align*}
            y_1 + x + y_2 &= (y_1 + x) + y_2 \\
            &= y_1 + (x + y_2) \\
            &= O + y_2 \\
            &= y_1 + O \\
            &\Rightarrow y_2 = y_1 
        \end{align*}
    \end{proof}
    \begin{proof}[Proof of theorem 3]
        We prove the theorem in a list:
        \begin{enumerate}
            \item \begin{align*}
                0x &= (0+0)x \\
                &= 0x + 0x \\
                0x + -(0x) &= 0x + (0x + -(0x)) \\
                O &= 0x + O \\
                O &= 0x
            \end{align*}
            \item \begin{align*}
                aO + aO &= a(O + O) \\
                &= aO \\
                aO + aO + (-aO) &= aO + (-aO) \\
                aO &= O 
            \end{align*}
            \item \begin{align*}
                (-a)x + ax &= (-a + a)x \\
                &= 0x \\
                &= O
            \end{align*}
            \item Suppose that $a=0$. Then by part (a), we have $ax = O$. If $a \neq 0$, then either $x = O$ or $x \neq O$. %lol, maybe we can assume that because we already know multiplicative properties of the reals, we can just fanangle a 'proof'
            \item Since we have $ax = ay$, we have
            \begin{align*}
                ax &= ay \\
                \frac1a(ax) &= \frac1a(ay) \\
                \left(\frac1aa\right)x &= \left(\frac1aa\right)y 
                x &= y
            \end{align*}
            \item asdfasdf 
        \end{enumerate}
    \end{proof}

    \section*{Homework}
    \begin{enumerate} % 3, 6, 11, 17, 20-24
        \item Determine if the set of all vectors $(x,y,z)$ in $V_3$ whose components satisfy the condition $x+y+z = 0$ is a subspace of $V_3$. If it is, compute its dimension.
        \item Determine if the set of all vectors $(x,y,z)$ in $V_3$ whose components satisfy the condition $x=y$ or $x=z$ is a subspace of $V_3$. If it is, compute its dimension.
        \item Determine if the set $S$ of polynomials of degree $\leq n$ satisfying the condition $f(0)=0)$ is a subspace of $P_n$. If it is, compute its dimension.
        \item Determine if the set $S$ of polynomials of degree $\leq n$ satisfying the condition $f$ is even is a subspace of $P_n$. If it is, compute its dimension.
        \item Determine if the set $S$ of polynomials of degree $\leq n$ satisfying the condition $f$ has degree $k < n$ or $f = 0$ is a subspace of $P_n$. If it is, compute its dimension.
        \item In the linaer space of all real polynomials $p(t)$, describe the subspace spanned by each of the following subsets of polynomials and determine the dimension of this subspace. 
        \begin{enumerate}
            \item $\{1,t^2,t^4\};$
            \item $\{t,t^3,t^5\};$
            \item $\{t,t^2\};$
            \item $\{1+t,(1+t)^2\};$
        \end{enumerate}
        \item In this exercise, $L(S)$ denotes the subspacew spanned by a subset $S$ of a linaer space $V$. Prove each of the statements (a) through (f).
        \begin{enumerate}
            \item $S\subseteq L(S)$.
            \item If $S \subseteq T \subseteq V$ and if $T$ is a subspace of $V$, then $L(S) \subseteq T$. This property is described by saying that $L(S)$ is the smallest subspace of $V$ which contain $S$.
            \item A subset $S$ of $V$ is a subspace of $V$ if and only if $L(S) = S$.
            \item If $S \subseteq T \subseteq V$, then $L(S) \subseteq L(T)$.
            \item If $S$ and $T$ are subspaces of $V$, then so is $S \bigcap T$.
            \item If $S$ and $T$ aer subsets of $V$, then $L(S \cap T) \subseteq L(S) \cap L(T)$.
            \item Give an example in which $L(S \cap T) \neq L(S) \bigcap L(T)$.
        \end{enumerate}
        \item Let $V$ be the linear space consisting of all real-valued functions defined on the real line. Determine whether each of the following subsets of $V$ is dependent or independent. Compute the dimension of the subspace spanned by each set.
        \begin{enumerate}
            \item $\{1, e^{ax}, e^{bx}\}, a \neq b$
            \item $\{e^{ax}, xe^{ax}\}$
            \item $\{1, e^{ax}, xe^{ax}\}$
            \item $\{e^{ax}, xe^{ax}, x^2e^{ax}\}$
            \item $\{e^x, e^{-x}, \cosh x\}$
            \item $\{\cos x, \sin x\}$
            \item $\{\cos^2 x, \sin^2 x\}$
            \item $\{1, \cos 2x, \sin^2 x\}$
            \item $\{\sin x, \sin 2x\}$
            \item $\{e^x\cos x, e^{-x}\sin x\}$
        \end{enumerate}
        \item Let $V$ be a finite-dimensional linear space, and let $S$ be a subspace of $V$. Prove each of the following statements.
        \begin{enumerate}
            \item $S$ is finite dimensional and $\dim S < \dim V$.
            \item $\dim S = \dim V$ if and only if $S = V$.
            \item Every basis for $S$ is part of a basis for $V$.
            \item A basis for $V$ need not contain a basis for $S$.
        \end{enumerate}
    \end{enumerate}
\end{document}